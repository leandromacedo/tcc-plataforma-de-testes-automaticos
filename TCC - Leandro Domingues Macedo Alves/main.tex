\documentclass[grad,numbers]{style/coppe}
\usepackage{amsmath,amssymb}
\usepackage{hyperref}
\usepackage[utf8]{inputenc}
\usepackage[brazil]{babel}
\usepackage[T1]{fontenc}
\usepackage{graphicx}
\usepackage{tabularx}
\usepackage{minted}
\usepackage{caption}
\usepackage{import}
\usepackage{indentfirst}

\setminted{
    linenos=true,
    autogobble,
}
\newenvironment{longlisting}{\captionsetup{type=listing}}{}

\makelosymbols
\makeloabbreviations

\begin{document}
  \title{Ambiente ágil de testes automáticos em nuvem para os sistemas web FENCE dos experimentos do CERN}
  \foreigntitle{}
  \author{Leandro}{Domingues Macedo Alves}
  \advisor{}{Carmen}{Lucia Lodi Maidantchik}{D.Sc.}
  \advisor{Prof.}{Flávio}{Luis de Mello}{D.Sc.}

  \examiner{}{Carmen Lucia Lodi Maidantchik}{D.Sc.}
  \examiner{Prof.}{Flávio Luis de Mello}{D.Sc.}
  \examiner{Prof.}{Jomar Gozzi}{D.Sc.}
  \examiner{Prof.}{Carlos José Ribas D'Avila}{D.Sc.}
  
  \department{ECA}
  \date{10}{2019}

  \keyword{Sistemas \textit{Web}}
  \keyword{Testes automáticos}
  \keyword{Integração Contínua}
  \keyword{\textit{CERN}}

  \maketitle

  \frontmatter
  
  \makecatalog
 
 % \chapter*{Agradecimentos}
    
  %\indent
  %*agradecimentos*
  
  \import{source/}{abstract.tex}

  \tableofcontents
  \listoffigures
  \printlosymbols
  \printloabbreviations

  \mainmatter

  \import{source/1-introducao/}{1.0-all.tex}
  
  \import{source/2-contextualizacao/}{2.0-all.tex}
  
  \import{source/3-fundamentacao/}{3.0-all.tex}

  \import{source/4-solucao/}{4.0-all.tex}

  \import{source/5-resultados/}{5.0-all.tex}

  \import{source/6-conclusao/}{6.0-all.tex}

  \backmatter
  \nocite{*}
  \bibliography{source/bibliografia/all}\label{bibliografia}
  \bibliographystyle{style/coppe-unsrt}
\end{document}
