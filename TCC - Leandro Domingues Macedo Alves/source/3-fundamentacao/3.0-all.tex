\chapter{Fundamentação}\label{cap_fundamentacao}

Com a constante procura na melhoria no processo de desenvolvimento de software, o grupo se esforça em adotar gradativamente valores ágeis, com ciclos de entrega mais curtos e garantia da qualidade de entrega. Esse esforço foi iniciado com uma pesquisa na literatura, principalmente na teoria de testes.
Uma visão geral do assunto é apresentada ao longo desta fundamentação, analisando o planejamento de testes e suas diversas categorias. A partir disso é feita uma divisão entre testes manuais e automáticos, enfatizando os testes automáticos. São discutidos com mais profundidade os testes de unidade e \emph{end-to-end}, e por fim são estudadas as possibilidades existentes de ter um ambiente distribuído de execução de testes com integração contínua.

\import{source/3-fundamentacao/}{3.1-aplicacao.tex}
\import{source/3-fundamentacao/}{3.2-teste-de-software.tex}
\import{source/3-fundamentacao/}{3.3-testes-unitarios.tex}
\import{source/3-fundamentacao/}{3.4-testes-ponta-a-ponta.tex}
\import{source/3-fundamentacao/}{3.5-integracao-continua.tex}