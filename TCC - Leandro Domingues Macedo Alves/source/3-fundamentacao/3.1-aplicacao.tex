\section{Aplicações web e metodologia ágil}
Com o início da popularização da internet a partir da concepção do \emph{World Wide Web} por Berners-Lee \cite{Berners-Lee} no próprio \emph{CERN}, o cenário de desenvolvimento e consumo de software passou por uma grande transformação. Navegadores web passaram a existir, cada vez mais poderosos e com cada vez mais capacidade de exibir conteúdo remoto prontamente. Dessa forma, tornou-se favorável o desenvolvimento de aplicações web.

Como salientado por Vora \cite{vora}, aplicações web tornam extremamente fácil para usuários fazerem uso do software desenvolvido, sendo necessário apenas que tal software seja hospedado em um servidor capaz de suprir a demanda. A atualização do software também se torna mais rápida, pois uma vez implementada a atualização no servidor, é possível garantir uniformidade para todos que o acessam. Do lado dos desenvolvedores também se torna uma tarefa mais simples, já que na maior parte dos casos, os navegadores são os que assumem a responsabilidade de tornar compatível o conteúdo remoto para diferentes plataformas. Deste modo, o programador não precisa se preocupar se sua aplicação irá funcionar para um cliente com um ambiente computacional específico, contanto que seja utilizado um navegador com uma versão suportada pelo software em questão. As aplicações web também permitem a redução em custos computacionais pois permite que a base de código seja dividida entre servidor e cliente, com o \emph{back-end} sendo o código executado pelo servidor e o \emph{front-end} possui o código que será processado nos navegadores dos usuários.

Por conta destas características, e do dinamismo presente no \emph{CERN}, desde o início do desenvolvimento de sistemas \emph{Glance} optou-se pela implementação de softwares em web. O resultado obtido com esta implementação foi positivo, e o crescimento sucessivo do número de sistemas proporcionou o aumento correspondente no número de desenvolvedores do grupo. A ampliação do grupo passou a motivar reflexões recentes sobre a forma na qual vem sendo desenvolvido software, e como aprimorar a comunicação. Estas reflexões levam em consideração o ambiente de desenvolvimento do grupo, marcado por alterações recorrentes de requisitos e urgência nas entregas. Isso evidencia a importância em adaptar continuamente o software a mudanças e com ciclos curtos de desenvolvimento, sendo estas características semelhantes com os valores e princípios empregados pelas metodologias ágeis existentes.

Apesar do destaque recente recebido pelo movimento ágil, este pensamento possui décadas de existência. Baseando-se na análise feita em Myers \emph{et al.} \cite{myers}, com o aumento da competição e interconectividade em todos os setores da economia durante os anos 90, as estratégias de negócios passaram a adotar soluções que reduzissem o tempo de lançamento de um produto sem afetar sua qualidade. Na área de software não foi diferente, e o processo de desenvolvimento tradicional não se mostrava tão satisfatório nesse novo ambiente competitivo, principalmente com a popularização da internet.

No início dos anos 2000, um grupo de desenvolvedores propôs um novo processo de trabalho, chamado de manifesto ágil. Esse documento continha uma nova filosofia, focada em clientes e funcionários ao invés de processos e hierarquias. O desenvolvimento ágil promove um processo interativo e incremental, centrado no cliente, incluindo mudanças durante o processo, e possuindo ciclos curtos de entrega. O método não é considerado eficiente se não obtiver a participação constante dos clientes durante o desenvolvimento. É importante a validação contínua destes clientes sobre a implementação progressiva do produto e se atende as expectativas.

Uma das abordagens de metodologia ágil mais adotadas na comunidade é a programação extrema, conhecida como \emph{Extreme Programming} ou \emph{XP}, e desenvolvida por Beck \emph{et al.} \cite{beck1}. Além da participação do usuário, o modelo \emph{XP} depende fortemente de testes unitários e de aceitação, e sua adoção é recomendada principalmente em cenários sujeitos a mudanças frequentes de especificações e que possuem um canal acessível para comunicação constante. O planejamento foca em entregáveis, chamados \emph{user stories} que definem os requisitos dos usuários. Estas \emph{user stories} podem então ser verificadas no final de um ciclo de entrega, por meio dos testes de aceitação do usuário. Dessa forma, se obtém retorno da opinião dos usuários mesmo durante o desenvolvimento, reduzindo as chances de um produto que não atenda as expectativas finais.

O desenvolvimento ágil também se estende a nível de código, pois é fundamental o esforço em escrever código que seja receptivo a mudanças, extensível, robusto e manutenível. A fim de alcançar estas metas um conjunto de princípios de \emph{design}, abreviado \emph{SOLID}, foi desenvolvido por Martin \cite{martin}. São cinco princípios teoricamente capazes de aprimorar a compreensão, manutenção e escalabilidade de código. Citando-os em ordem, o princípio da responsabilidade única define que um módulo deve ter apenas um motivo para mudar; o princípio do aberto-fechado determina que as entidades de software devem ser abertas para ampliação mas fechadas para modificação; o princípio da substituição de \emph{Liskov} diz que os subtipos devem ser substituíveis pelos seus tipos de base; o princípio da segregação de interfaces defende o desenvolvimento de interfaces refinadas que são específicas para o cliente; e finalmente o princípio da inversão de dependências estabelece que só se deve depender de abstrações, se desvencilhando de classes concretas. Estes princípios são amplamente utilizados pela comunidade no desenvolvimento de aplicações em orientação a objetos.

Outro conceito importante é o do teste ágil, mais conhecido como \emph{agile testing} \cite{myers}. O teste ágil é uma forma colaborativa de teste, na qual desenvolvedores, testadores e clientes fazem parte do planejamento, implementação e execução do plano de testes. Os clientes contribuem para os testes de aceitação com casos de uso e expectativas de funcionalidade, enquanto os desenvolvedores colaboram com os testadores implementando o ambiente de testes e automatizando-os quando possível. Como nas metodologias ágeis os ciclos de desenvolvimento são curtos, o tempo se torna um recurso valioso, e por este motivo a filosofia do teste ágil apoia a implantação de testes automáticos. Este foi um dos motivacionais para o aprofundamento deste trabalho em testes automáticos.