\section{Justificativa}

Até o início deste trabalho, todos os testes em sistemas eram realizados de forma manual pelo grupo de desenvolvedores da UFRJ. Estes sistemas web fazem parte da colaboração internacional entre a UFRJ e o \emph{CERN} para gerenciar publicações, equipamentos e pessoas. A validação e a verificação destes softwares fazem uso de planos de testes, que contêm as regras dos sistemas para serem testadas pelos desenvolvedores. Para a garantia de qualidade do software, é necessário que estas verificações sejam executadas frequentemente, o que pode se tornar algo exaustivo e consumir uma quantidade significativa de tempo. Dessa forma, almejou-se otimizar o processo de verificação, com a implementação de testes automáticos para estes sistemas. O principal objetivo destes testes é garantir o comportamento esperado minimizando a necessidade de intervenção humana para verificar cada etapa. Assim que codificados, estes testes podem ser executados de maneira contínua e bem estruturada, comparando os resultados esperados com os obtidos. Eles podem periodicamente verificar a condição de um sistema web assegurando suas funções, desde que cobertas pelos próprios testes.