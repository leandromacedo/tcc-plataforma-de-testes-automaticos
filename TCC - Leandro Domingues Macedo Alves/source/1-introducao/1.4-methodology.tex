\section{Metodologia}
 O início deste trabalho se deu com um embasamento teórico a partir de artigos e livros no assunto de testes de software. Depois foi proposto a implementação de um protótipo baseado no conceito de testes de interface. O objetivo deste protótipo estava na realização de testes ponta a ponta básicos, popularmente conhecidos de testes \emph{end-to-end}, os quais realizavam comandos simples no navegador web, como por exemplo a verificação se um membro era exibido na interface quando buscado por seu nome. Tais testes foram escritos utilizando um interpretador de código capaz de realizar as respectivas ações no navegador.

 Após esta prova de conceito, foi analisado o impacto que os testes automáticos teriam na qualidade dos sistemas providos pelo grupo, e quais seriam os desafios em acomodá-los no processo de trabalho dos desenvolvedores. Dessa forma, a proposta do projeto foi expandida para a implementação de uma plataforma capaz de dar as ferramentas necessárias para o desenvolvedor implantar testes unitários e testes \emph{end-to-end} com mais facilidade. Esta plataforma também teria que estar apta a administrar estes testes de forma automática, executando-os conforme o ciclo de desenvolvimento.

Aperfeiçoando o protótipo inicial para dar mais funcionalidade, foi desenvolvido um \emph{framework} voltado para a construção e compartilhamento de testes \emph{end-to-end}. A ideia principal destes testes é verificar o sistema sob a perspectiva dos usuários, comparando os resultados obtidos diretamente com os esperados pelos requisitos. Com estes resultados, os desenvolvedores podem avaliar a necessidade de alterações antes de entregar uma nova versão do sistema em produção.

Posteriormente foi desenvolvido um segundo \emph{framework} que se torna o responsável pela construção de testes unitários de código contido no servidor. Diferentemente dos testes \emph{end-to-end}, os testes unitários realizam verificações a um nível modular e isolado, conferindo o comportamento de pequenos trechos de código. Este tipo de teste auxilia em encontrar problemas logo no início do ciclo de desenvolvimento e também permite ao programador reescrever o código do sistema posteriormente, garantindo que o módulo ainda funcione corretamente.

A terceira parte do trabalho foca no desenvolvimento de um ambiente baseado em sistemas distribuídos para execução destes testes. Fazendo uso de máquinas virtuais disponibilizadas pelo \emph{CERN}, foi implementado um ambiente de integração e entrega contínua de software. Quando o desenvolvedor prepara uma nova versão de um sistema, uma série de tarefas são automaticamente executadas na tentativa de certificar que problemas não ocorrerão no ambiente de produção. Estas tarefas incluem compilação, testes unitários, testes \emph{end-to-end} e simulação de instalação do software em ambiente de produção.

Durante a evolução deste trabalho foram agregados conhecimentos sobre práticas ágeis que poderiam contribuir para o processo de desenvolvimento de software do grupo de desenvolvedores. Ferramentas e rotinas foram adotadas, com o intuito de melhorar a comunicação interna, aproximar os usuários e prover ciclos curtos de entrega.

É importante citar também a colaboração de outras pessoas do grupo que possibilitaram a expansão deste trabalho. O processo de entendimento dos problemas contou principalmente com a participação de programadores que desenvolvem sistemas para as colaborações \emph{ATLAS}, \emph{LHCb} e \emph{ALICE}. Reuniões recorrentes foram realizadas a fim de entender com profundidade o problema e discutir as diferentes estratégias disponíveis para tentar solucioná-lo. Estas pessoas também contribuíram para o desenvolvimento da plataforma deste trabalho, assim como na implantação de testes automáticos.

O projeto também envolveu profissionais e estudantes da UFRJ sendo utilizada uma comunicação principalmente através de e-mails, reuniões remotas por conferência, ferramenta de mensagens \emph{Slack} e pelo sistema de gerenciamento de software, \emph{JIRA}.
