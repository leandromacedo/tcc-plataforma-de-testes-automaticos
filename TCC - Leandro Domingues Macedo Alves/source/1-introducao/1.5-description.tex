\section{Descrição}
O conteúdo deste documento está dividido em 6 capítulos. O capítulo atual descreve o tema do trabalho, o problema no qual ele se aplica, o seu objetivo e a metodologia por trás dele.

Uma breve contextualização é feita no capítulo \ref{cap_contextualizacao}, explicando a conjuntura da colaboração com o \emph{CERN} e as características específicas deste ambiente que motivaram esta dissertação.

O capítulo \ref{cap_fundamentacao} trata da fundamentação teórica, apresentando principalmente a teoria de testes e os autores que influenciaram a elaboração desta atividade.

As especificações da solução proposta são definidas no capítulo \ref{cap_solucao}. É aprofundada a explicação da arquitetura da plataforma desenvolvida, explicando o funcionamento, suas tecnologias e casos de uso. Por fim é falado sobre a solução de integração e de entrega contínua adotada.

No capítulo \ref{cap_resultados} são apresentados os resultados obtidos com a implantação da plataforma para projetar, implementar e integrar testes automáticos no processo de trabalho. O capítulo \ref{cap_conclusao} é reservado à conclusão.
