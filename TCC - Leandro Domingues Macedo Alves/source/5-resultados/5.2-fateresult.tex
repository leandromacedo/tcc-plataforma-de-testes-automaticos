\hypertarget{fate-resultados}{%
\section{Validação de usuário com testes ponta a ponta}\label{fate-resultados}}

Dada a complexidade em realizar testes \emph{e2e}, suas estatísticas possuem uma ordem de grandeza inferior a dos testes unitários. Por meio do \emph{Fate}, nos últimos 10 meses foram desenvolvidos 38 \emph{test cases} contidos em 5 \emph{test suites} para os sistemas \emph{Alice Membership} e \emph{Atlas Central Equipment System}. Estes testes avaliam 5 interfaces entre os dois sistemas, levando atualmente em média 3 minutos para execução de todos os testes. Essa duração era consideravelmente maior no passado, podendo a chegar a mais de 15 minutos em uma execução, mas dado os esforços em paralelizar e otimizar as máquinas hospedeiras, este tempo foi reduzido. Atualmente 6 instâncias de máquinas são requisitadas, com uma instância \emph{hub} administrando os testes para as 5 instâncias \emph{node} os executarem no navegador \emph{Chrome}. Este número de instâncias foi escolhido pois é proporcional ao número de testes atualmente existentes.

Como exemplos de testes \emph{e2e} desenvolvidos pelo grupo, é válido mencionar verificações de interações como acoplamento e desacoplamento de equipamentos, submissão de novos contratos, busca de equipamentos, e movimentações de \emph{racks}. Diferentemente da natureza atômica dos testes unitários, os testes \emph{e2e} são capazes de avaliar o comportamento completo de uma funcionalidade requisitada.

Nos últimos 4 meses, com o início das \emph{pipelines}, foram feitas mais de 500 execuções de testes \emph{e2e}, acumulando um total de quase 15 mil \emph{test cases} executados em uma duração de mais de 37 horas. Destas execuções, 79\% obtiveram sucesso, e 13\% indicaram pelo menos uma inconsistência. O motivo do valor mais elevado na taxa de erros em relação aos testes unitários é o fato da maior sensibilidade presente nos testes \emph{e2e}. Como dito anteriormente, testes avaliam o software como um todo, tanto em quesitos funcionais como não funcionais, e, portanto, instabilidades na infraestrutura ou falta de robustez no próprio desenvolvimento do teste condicionam à falha. Entretanto, da mesma forma que a chance de falsos positivos de erro é maior do que a dos testes unitários, os testes \emph{e2e} conseguem ter um alcance maior no encontro de inconsistências para uma mesma equivalência a ser verificada por testes unitários.

Como aspiração para os próximos passos, se mostra imprescindível o aumento do desenvolvimento de testes \emph{e2e} dentro do grupo, assim como no caso dos testes unitários. As ferramentas disponíveis dentro do \emph{Fate} facilitam o desenvolvimento dos testes, entretanto é necessário a familiarização com o \emph{framework}. É de interesse também implementar abstrações no \emph{framework} para aumentar a estabilidade dos testes desenvolvidos pelo grupo. Por exemplo, se é de conhecimento que para uma mesma interação ocorrem comportamentos distintos para \emph{Chrome} e \emph{Firefox}, é interessante encapsular e tratar estes comportamentos em um módulo para ser usado pelo desenvolvedor do teste, evitando que o próprio tenha esta tarefa adicional.
