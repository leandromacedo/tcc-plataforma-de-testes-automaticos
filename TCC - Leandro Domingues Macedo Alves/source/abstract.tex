\begin{abstract}
Atualmente o grupo de desenvolvedores da colaboração entre UFRJ e \emph{CERN} disponibilizam mais de 30 sistemas web para atender mais de 2 mil usuários por mês. É necessária a garantia do funcionamento correto destes sistemas, e o tema deste projeto de conclusão de curso está relacionado à implementação de uma plataforma de testes automáticos para verificação destes sistemas.

Em linhas gerais, objetiva-se o desenvolvimento de abstrações para facilitar a escrita de testes automáticos, juntamente com uma solução em integração contínua para administrar a execução destes testes. Estas abstrações por sua vez foram desenvolvidas para escrita de testes de unidade e de ponta a ponta, oferecendo funcionalidades para verificar os sistemas que compartilham o \emph{framework} \emph{Fence}, desenvolvido pelo grupo para geração de interfaces web. A implementação da integração contínua para estes testes permitiu a execução automática de mais de 150 mil testes e mais de 800 inconsistências foram encontradas antecipadamente, motivando a escrita de mais testes utilizando a plataforma.
\end{abstract}

%\begin{foreignabstract}
%  *abstract english*
%\end{foreignabstract}