\hypertarget{trabalhos-futuros}{%
\section{Trabalhos Futuros}\label{trabalhos-futuros}}

Em relação a trabalhos futuros para o \emph{FUnit}, há principalmente o interesse em aperfeiçoá-lo com um melhor suporte a testes de integração. Atualmente é possível a realização destes testes, mas há oportunidades para otimização, como a criação de abstrações para facilitar testes em \emph{REST API}, que se trata de um padrão de arquitetura utilizado para implementação de serviços web. Em relação a testes unitários, algumas funcionalidades extras poderiam ser desenvolvidas no futuro. Por exemplo, é desejável ter a possibilidade de múltiplas condições de entrada para um único \emph{test case}, de forma que possa ser executado repetidamente para cada condição. Seria interessante também o \emph{framework} ter capacidade de desenvolver \emph{mocks} de outras classes além do tradicional \emph{mock} que representa a própria classe a ser testada, algo que é especialmente útil para métodos que fazem chamadas a outras instâncias de classe. Outra ideia é a extensão de testes unitários para além da linguagem \emph{php}, estendendo a implementação dos testes para a linguagem \emph{javascript}.

Como aspiração para os próximos passos para o \emph{Fate}, se mostra imprescindível o aumento do desenvolvimento de testes \emph{end-to-end} dentro do grupo, assim como no caso dos testes unitários. As ferramentas disponíveis dentro do \emph{Fate} facilitam o desenvolvimento dos testes, entretanto é necessária a familiarização com o \emph{framework}. É de interesse também implementar abstrações no \emph{framework} para aumentar a estabilidade dos testes desenvolvidos pelo grupo. Por exemplo, se é de conhecimento que para uma mesma interação ocorrem comportamentos distintos para \emph{Chrome} e \emph{Firefox}, é interessante encapsular e tratar estes comportamentos em um módulo para ser usado pelo desenvolvedor do teste, evitando que o próprio tenha esta tarefa adicional.

Finalmente para o \emph{Fence CID} seria interessante incrementar a integração contínua com mais tarefas a serem realizadas, como testes de validação de sintaxe e verificação de vulnerabilidade em dependências. É almejado também a possibilidade de estender a entrega contínua para o ambiente de produção, tarefa que ainda é feita manualmente dada sua sensibilidade.