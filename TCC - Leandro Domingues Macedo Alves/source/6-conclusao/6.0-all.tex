\chapter{Conclusão}\label{cap_conclusao}

Presente no maior experimento científico já construído, este trabalho faz parte da colaboração entre UFRJ e \emph{CERN}, com o desenvolvimento de aplicações web que vão desde o monitoramento de equipamentos até a gerência de membros, atendendo as necessidades dos experimentos \emph{ATLAS}, \emph{ALICE} e \emph{LHCb}. Atualmente 27 sistemas com mais de 3 mil usuários foram implementados pelo grupo, fazendo uso do \emph{framework} \emph{Fence} para reutilização de código e rápida resposta em relação a mudanças.

Com o aumento de sistemas, foi levado em consideração a possibilidade de tentar melhorar a qualidade do produto sem prejudicar o tempo de entregas. Utilizando a filosofia de metodologias ágeis como \emph{Extreme Programming}, diversos processos e métodos foram implementadas no grupo de desenvolvedores com os objetivos de melhorar a comunicação, aumentar a transparência e principalmente prover um software confiável. Nesse último ponto, que é o tema principal deste trabalho, o maior esforço foi realizado na área de testes automáticos e em \emph{devops}, com o desenvolvimento de dois \emph{frameworks}, \emph{Fate} e \emph{FUnit}, e de um ambiente de integração e entrega contínua, o \emph{Fence CID}.

Para as metas de comunicação e transparência, diversas ferramentas e rotinas foram apresentadas aos desenvolvedores do grupo. \emph{Merge requests}, \emph{Slack}, \emph{Confluence}, \emph{Linters}, \emph{Lean Testing} e \emph{stand-up meetings} estão entre elas, destacando-se o as discussões produtivas entre desenvolvedores nas revisões de código do \emph{merge request} e a padronização e aprendizado de código proporcionado pelos \emph{linters}. Todas essas ideias possuíram bons resultados, excetuando-se possivelmente o \emph{Lean Testing}, de modo que uma nova estratégia para o desenvolvimento de plano de testes possa ser necessária.

Em relação à confiabilidade de software, os testes automáticos possuíram resultados significativos. Testes unitários são rápidos, modulares e fáceis de escrever, considerando-se também os ganhos com documentação e orientação a boas práticas de código. Já os testes \emph{e2e} conseguem avaliar uma funcionalidade por completo, na mesma perspectiva que o usuário final, e fazer um verificação complementar a dos testes unitários. Apesar de serem mais custosos e demorados, estes tipos de teste compensam para projetos de médio e longo prazo, com a capacidade de avaliar o comportamento do software em diferentes ambientes operacionais. Apesar das facilidades de ambos \emph{frameworks} \emph{Fate} e \emph{FUnit} para o desenvolvimento destes testes, o número ainda é baixo em relação a base de código existente, e portanto há a urgência na adoção da prática de desenvolvimento de testes enquanto ocorre o desenvolvimento, com o uso do \emph{FUnit} para testes unitários durante a escrita do código, e o uso do \emph{Fate} quando a funcionalidade estiver finalizada e for apresentada para o usuário.

O \emph{Fence CID} tornou possível a orquestração da verificação e entrega de software pelo grupo. Compilando, realizando testes e por fim executando o \emph{deploy}, as \emph{pipelines} concederam tempo e garantia para os desenvolvedores com a execução automática destas tarefas. A possibilidade de paralelizar tarefas com um ambiente de sistemas distribuídos em \emph{Docker} também trouxe benefícios em relação a uma possível escalabilidade de sistemas no futuro.

Testes automáticos com apoio de metodologias ágeis vêm ganhando cada vez mais adesão pela comunidade de programadores. A proposta inicial deste trabalho era a implementação de uma plataforma de testes automáticos que trouxesse melhorias para a confiabilidade do software. Apesar de ainda possuir uma quantidade de testes automáticos pequena em relação a base de código existente, a plataforma de testes implementada conseguiu atingir o objetivo, possuindo resultados significativos na localização antecipada de inconsistências. Adicionalmente, a plataforma possibilitou a redução de tempo gasto pelos desenvolvedores com a entrega de software, e as práticas ágeis adotadas pelo grupo durante o desenvolvimento desta plataforma contribuíram para a comunicação e organização dos desenvolvedores.
